\documentclass[a4paper, 14pt]{extreport}
\usepackage[utf8]{inputenc}
\usepackage[T2A]{fontenc}
\usepackage[russian]{babel}
\usepackage{lscape}
% поля:
\usepackage[left=2.5cm, right=1.5cm, vmargin=2.5cm]{geometry}
\usepackage{indentfirst} % отделять первую строку раздела абзацным отступом
\setlength\parindent{5ex}

% настройка разделов
\usepackage{titlesec}

\titleformat{\chapter}[block]
{\filcenter}
{\thechapter}
{1em}
{\MakeUppercase}{}

\titleformat{\section}
{}
{\thesection}
{1ex}{}

\titleformat{\subsection}
{}
{\thesubsection}
{1ex}{}

\setcounter{secnumdepth}{4}
\titleformat{\subsubsection}[runin]
{}
{\thesubsubsection}
{1ex}{}[.]

% настройка Содержания
\addto{\captionsrussian}{\renewcommand*{\contentsname}{Содержание}}
\usepackage{titletoc}
\dottedcontents{chapter}[1.6em]{}{1.6em}{1pc}
\usepackage[hidelinks]{hyperref} % гиперссылки в содержании

% Обработка таблиц
\usepackage{tabularx}
\usepackage{datatool}

\usepackage{lastpage} % ссылка на последнюю страницу документа

\usepackage{caption} % подписи к рисункам в русской типографской традиции
\DeclareCaptionFormat{GOSTtable}{#2#1\\#3}
\DeclareCaptionLabelSeparator{fill}{\hfill}
\DeclareCaptionLabelFormat{fullparents}{\bothIfFirst{#1}{~}#2}
\captionsetup[table]{
	format=GOSTtable,
	font={footnotesize},
	labelformat=fullparents,
	labelsep=fill,
	textfont=bf,
	justification=centering,
	singlelinecheck=false
}

\begin{document}
\selectlanguage{russian} % основной язык документа
\begin{titlepage}                                                         
	\newpage                                                                        
	\begin{center}                                                        
		Публичное акционерное общество  <<Газпром>>\\
			Общество с ограниченной ответственностью\\
			<<Научно-исследовательский институт экономики и организации управления в газовой промышленности>>\\
			(ООО <<НИИгазэкономика>>)                             
                                                            
		\vspace{3em}                                                          
                                                     
		\begin{tabular}{p{0.6\textwidth}@{}l l}
			УДК & \\
			№ госрегистрации & \\
			СОГЛАСОВАНО & УТВЕРЖДАЮ \\
			Начальник Департамента & Генеральный директор \\
			ПАО <<Газпром>> & ООО <<НИИгазэкономика>> \\
			 & канд. тех. наук \\
			  & \\
			\makebox[3cm]{\hrulefill} В.А. Михаленко & \makebox[3cm]{\hrulefill} Н.А. Кисленко \\
			<<\makebox[1cm]{\hrulefill}>> \makebox[2cm]{\hrulefill} 2020 г. & <<\makebox[1cm]{\hrulefill}>> \makebox[2cm]{\hrulefill} 2020 г. \\
		\end{tabular}                                                    
	\end{center}                                                          
	
	\vspace{1.2em}                                                        
	
	\begin{center}                                                        
		%\textsc{\textbf{}}                                     
		\Large ОТЧЕТ \linebreak                                  
		об оказании услуг
	\end{center}                                                          
	
	\vspace{1em}        
		\begin{tabularx}{0.95\textwidth}{|>{\hsize=.55\hsize\linewidth=\hsize}X|
				>{\hsize=1.45\hsize\linewidth=\hsize}X|}
			\hline
		№, дата договора & № 1357/18 от 26.06.2018 г.\\
		\hline
		Наименование\newline договора & Прогноз технически возможной производительности участков газотранспортной системы на 10 лет.\\
		\hline
		№, наименование\newline этапа & Этап 2. Прогноз технически возможной производительности участков газотранспортной системы на 10 лет.
\newline
		Книга 1. Общая пояснительная записка.
\\
		\hline
		Наименование\newline результата & Прогноз технически возможной производительности участков газотранспортной системы на 10 лет.\\
		\hline
	\end{tabularx}                                                  
	                                                        
	\vspace{2em}                                                                    


	\begin{tabularx}{0.9\textwidth}{>{\hsize=.5\hsize\linewidth=\hsize}X
			>{\hsize=2.5\hsize\linewidth=\hsize}r}
			\noindent\parbox[b]{\hsize}{Руководитель работы\newline заместитель директора центра,\newline канд. физ.-мат. наук} & \makebox[3cm]{\hrulefill} И.О. Стурейко\\
	\end{tabularx}       
                                  

	\vspace{\fill}                                                    
	
	\begin{center}                                                        
		Москва, 2020                                                                
	\end{center}                                                          
	
\end{titlepage}

\setcounter{page}{2} % начать нумерацию с номера два

\chapter*{список исполнителей}
	\begin{tabularx}{0.9\textwidth}{>{\hsize=.5\hsize\linewidth=\hsize}X
		>{\hsize=2.5\hsize\linewidth=\hsize}l}
	\noindent\parbox[b]{\hsize}{Руководитель работы\newline заместитель директора центра,\newline канд. физ.-мат. наук} & \makebox[3cm]{\hrulefill} И.О. Стурейко\\
	 & \\
	 & \\
	\noindent\parbox[b]{\hsize}{Заместитель заведующего отделом} & \makebox[3cm]{\hrulefill} Р.Я. Грыб\\
\end{tabularx} 

\chapter*{Реферат}
Отчет содержит \pageref{LastPage} страниц.

\begin{Large}
	твп, производительность, пропускная способность
\end{Large}

Целью работы является проведение расчетов и формирование прогноза технически возможной производительности существующих участков газотранспортной системы на десятилетний период, в том числе с учётом планируемых работ по капитальному ремонту и реконструкции.

Результатом работы является прогноз технически возможной производительности участком СМГ ГТС ПАО <<Газпром>>

%\tableofcontents

\chapter{Исходные данные}
%\input{initial_data.tex}
\chapter{Прогноз ТВП на 10-и летний период}
%\input{retro.tex}
%\input{prognoz.tex}
\chapter{Ранжирование участков ГТС с учётом планируемой загрузки}
%\input{range.tex}
\chapter{Прогноз СЦП 2-го уровня, связанных с показателем ТВП}
%\input{scp.tex}


%Текст. Как показано в ~табл. \ref{table_1} на странице \pageref{table_1} 


%	\begin{landscape}
%		\DTLsetseparator{|}      % разделитель ячеек
%		\DTLloaddb[noheader, keys={id,name,2012,2013,2014,2015,2016,2017,2018}, headers={id,Наименование,2012,2013,2014,2015,2016,2017,2018}]{satellites}{TVP.csv}
%		
%		\begin{table}
%			\caption{Ретроспектива ТВП}
%			\label{table_1}
%			\centering
%			\DTLdisplaydb{satellites}
%		\end{table}
	
%	\end{landscape}

	
%	Как изложено в таблице \ref{table_1}
	
	Дальнейший текст

\chapter{Приложения}
\appendix
\titleformat{\section}[display]
{\normalfont\Large\bfseries}
{\centering Приложение\ \thesection\\}
{0pt}{\Large\centering}
\renewcommand{\thesection}{\Asbuk{section}}
%\input{tvps2019}

%\renewcommand\appendixname{Приложение}
%\makeatletter
%\def\redeflsection{\def\l@section{\@dottedtocline{1}{1.5em}{7.8em}}}
%\renewcommand\appendix{\par
%	\setcounter{section}{0}%
%	\setcounter{subsection}{0}%
%	\def\@chapapp{\appendixname}%
%	\addtocontents{toc}{\protect\redeflsection}
%	\def\thesection{\appendixname\hspace{0.2cm}\@arabic\c@section}}
%\makeatother
%\input{tvps2019}
\end{document}
